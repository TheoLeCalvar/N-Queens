\documentclass[10pt,a4paper]{report}
\usepackage[utf8]{inputenc}
\usepackage[francais]{babel}
\usepackage[T1]{fontenc}
\usepackage{amsmath}
\usepackage{amsfonts}
\usepackage{amssymb}

\usepackage{algorithm}
\usepackage{algpseudocode}


\usepackage{graphicx}
\usepackage{fancyvrb}

\renewcommand{\thesection}{\arabic{section}}
\newcommand{\HRule}{\rule{\linewidth}{0.5mm}}
\VerbatimFootnotes


\begin{document}

\begin{titlepage}
\begin{center}

%\def\authors#1{\def\@authors{#1}}
%\newcommand{\printauthors}{\@authors}

% Upper part of the page. The '~' is needed because \\
% only works if a paragraph has started.
%\includegraphics[width=0.75\textwidth]{images/FileHub}~\\[2cm]

% \textsc{\LARGE UFR Sciences d'Angers}\\[1.5cm]

% \textsc{\huge FileHub}\\[1cm]

% Title
\HRule \\[0.4cm]
{ \huge \bfseries probleme des n dames \\[0.4cm] }

\HRule \\[1.5cm]

% Author and supervisor
\begin{minipage}{0.4\textwidth}
\begin{flushleft} \large
\emph{Auteurs:}\\
\textsc{Frémont} Alexandre\\
\textsc{Le Calvar} Théo\\

\end{flushleft}
\end{minipage}
\begin{minipage}{0.4\textwidth}
\begin{flushright} \large
\emph{Référent:} \\
\textsc{Lesaint} David \\
\textsc{Hao} Jin-Kao 
\end{flushright}
\end{minipage}

\vfill

% Bottom of the page
{\large Février 2015}

\end{center}
\end{titlepage}

\tableofcontents

\newpage

% \section{Résumé du projet}
\section{Résumé du projet}

Le problème des N-Dames est un extension du problème des 8 dames, son but est de placer 8 dames d'un jeu d'échec sur un échiquier de façon à ce qu'aucune ne soit menacée.
Pour notre problème nous ne considérons non plus un échiquier de $8*8$ cases mais sur un échiquier arbitrairement grand. On peut démontrer que pour tout $n >= 4$ il existe au moins une solution.

Dans ce rapport nous présenterons les algorithmes et les structures mis en place pour résoudre l'approche programmation par contrainte de ce problème.

Ce projet a été développé en C car ce langage nous permetait un contrôle fin de la mémoire, point particulièrement important.
A cause du fait que nous voulions avoir un programme proposant des algorithmes de résolutions utilisant la programmation par contrainte et la recherche locale nous avons dû mettre en place une structure de données commune.
Nous avons mis en place deux algorithmes, le backtracking et le forward checking.


% \section{structure de données}
\section{Structure de données}

Pour representer un échiquier nous avons décider d'utiliser un tableau a une dimension où chaque case représente l'emplacement d'une dame.
Ainsi l'indice du tableau correspond à la colonne de l'échiquier et la valeurs correspondant à la ligne.
Il est possible d'intervertir la significations des indices et des valeurs, les deux solutions représentent deux échiquiers symétriques par la diagonale.
Cette représentation présente l'immense avantage d'éliminer par construction les conflits sur les lignes et les colonnes (si on fait attention à ne pas introduire deux fois la même valeur dans le tableau),
limitant donc les conflits aux diagonales.

De plus, parce que chaque méthode de résolution nécessitait des informations complémentaires (gestion des domaines de valeurs pour le forward checking, maintient des conflits pour la recherche locale),
nous avons choisi de réduire la structure d'échiquier à son strict minimum et que chaque méthode apporterait ses structures complémentaires.

Ainsi le forward checking utilise un tableau de bitsets pour représenter les domaines, le backtracking lui tient juste à jour un bitset contenant les lignes disponibles.


% \section{backtrack}
\section{Backtracing}

Le backtracking est un algorithme générique 

% \section{forward checking}
\include{forward}

% \section{recherche local}
\section{la recherche local}

Pour la recherche local nous avons mis en place deux algoritme de descente strict.
Dans notre premier algoritme on place les reines sur l'échiquier de maniere aleatoire , en évitant les conflict de reines entre les lignes et les colonnes, donc les reines ne peuvent être en conflit que avec d'autres reines qui sont sur les mêmes diagonales.
Ensuite nous avons crées une structure de donnée pour retenir le positionnement des reines sur les diagonales. %pseudo code ?
Par la suite notre algoritme calcul le nombre de conflit qu'il y a dans l'échiquier pour chaque diagonales et en fait la somme, Apres cela l'algoritme selection au hasard une reine en conflict avec une autre reines et effectue un swap (echange de position entre les reines) avec une autre reine selection aléatoirement, une fois le swap effectué le programme vérifie que le nouveaux nombre de conflit calculé est égal ou inférieur au précedent nombre de conflit puis il recommence jusqu'a ce que le nombre de conflit soit nul, si le nombre de conflit generé par le swap est supérieur à l'ancien dans ce cas on annule le swap et l'algorithme continue.  
%pseudo code
Notre second algorithme de recherche local reprend les base du premier mais il utilise un systeme de bitset pour retenir les diagonales et les conflits des diagonales se systeme est plus rapide que les simples tableau normale.  

\end{document}
